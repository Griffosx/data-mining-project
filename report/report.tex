\documentclass[twoside,11pt]{article}

% Any additional packages needed should be included after jmlr2e.
% Note that jmlr2e.sty includes epsfig, amssymb, natbib and graphicx,
% and defines many common macros, such as 'proof' and 'example'.
%
% It also sets the bibliographystyle to plainnat; for more information on
% natbib citation styles, see the natbib documentation, a copy of which
% is archived at http://www.jmlr.org/format/natbib.pdf

\usepackage{jmlr2e}
\usepackage{lipsum}
\usepackage{amsmath}
\usepackage{hyperref}
\usepackage{algorithm,algorithmic}

\setcitestyle{square}

% Definitions of handy macros can go here

\newcommand{\dataset}{{\cal D}}
\newcommand{\fracpartial}[2]{\frac{\partial #1}{\partial  #2}}

\renewcommand{\algorithmicrequire}{\textbf{Input:}}
\renewcommand{\algorithmicensure}{\textbf{Output:}}

% Heading arguments are {volume}{year}{pages}{submitted}{published}{author-full-names}
\jmlrheading{}{2024}{}{}{}{Davide Giuseppe Griffon}

% Short headings should be running head and authors last names

\ShortHeadings{Predicting Air Pollution Levels in India}{Griffon}
\firstpageno{1}

\begin{document}

\title{Predicting Air Pollution Levels in Five Major Indian Cities}

\author{\name Aleksandr J. Smoliakov \email your.email@stud.mif.vu.lt \\
  \name Danial Yntykbay \email your.email@stud.mif.vu.lt \\
  \name Davide Giuseppe Griffon \email davide.griffon@stud.mif.vu.lt \\
  \addr Data Science study programme\\
       Faculty of Mathematics and Informatics}

\editor{Jurgita Markevi\v{c}i\={u}t\.{e}}

\maketitle


\begin{abstract}

In 2022, the United Nations recognized the right to a clean environment, including clean air, as a fundamental human right, underscoring the crucial link between environmental quality and human health. Air pollution ranks among the most pressing global health threats, causing an estimated 3 to 9 million deaths annually \citep{owid-data-review-air-pollution-deaths}, with India accounting for a substantial share \citep{Dey2020}. The country exemplifies this crisis, with multiple cities consistently listed among the world's most polluted. India's air pollution stems from a combination of natural and anthropogenic sources, emitting harmful pollutants such as carbon monoxide (CO) and particulate matter (PM\textsubscript{10} and PM\textsubscript{2.5}). Rapid expansion in industrial, power, and transportation sectors, coupled with both planned and unplanned urbanization, has significantly elevated ambient air pollution levels across the country \citep{gordon2018air}. The increasing number of automobiles and reliance on coal-based power generation are projected to further deteriorate air quality in the coming decade. Beyond human health implications, air pollution has severe consequences for economic productivity and climate change \citep{Yarlagadda2022}. Addressing this challenge requires a comprehensive understanding of the factors contributing to urban air pollution.

This project aims to develop a predictive model for air pollution levels in five major Indian cities — Bengaluru (Bangalore), Delhi, Hyderabad, Jaipur, and Mumbai — using a multivariate regression approach. The model will incorporate a wide array of predictors, including geographic location, seasonal variations, meteorological data, temperature, and possibly socio-economic factors. By evaluating various modeling techniques and analyzing data correlations, the study seeks to identify the most effective methods for accurately forecasting urban air pollution levels.

This study provides a practical application of data mining techniques within an academic setting, demonstrating the process of developing and evaluating predictive models using real-world environmental data. The findings will contribute to the students' understanding of data analysis in the context of air pollution and serve as a foundation for further exploration in this field.
  
\end{abstract}


\newpage

% ---------------------------------------------------------------------------------------
% ---------------------------------------------------------------------------------------
% ---------------------------------------------------------------------------------------


\section{Data Sources}

This project utilizes two primary sources of data to analyze air pollution levels across five major Indian cities:

\begin{itemize}
    \item \textbf{Air Quality Data in India}: Available at \href{https://www.kaggle.com/datasets/rohanrao/air-quality-data-in-india}{Kaggle}. This dataset provides hourly measurements of various air pollutants and particulate matter. The data is collected from multiple weather stations located within each of the five major cities. Recognizing these larger cities may have several monitoring stations to capture spatial variability in air quality, we aggregate the pollutant levels by averaging the measurements from all stations within a city for each hour. This averaging process ensures that the data represents the overall air quality of the city rather than isolated monitoring points.

    \item \textbf{Historical Weather Data for Indian Cities}: Available at \href{https://www.kaggle.com/datasets/hiteshsoneji/historical-weather-data-for-indian-cities?fbclid=IwY2xjawGJ4RpleHRuA2FlbQIxMAABHWvTrmK5HZhuOtOFaGmaStWQzwC0LiOHtK08q87gTHdQ36p_YVT9hVDUlg_aem_A0W4rpnHatpUfV7CUXfkkA}{Kaggle}. This dataset includes hourly weather-related features for the same set of cities, encompassing over 20 variables such as precipitation (mm), wind speed, temperature, humidity, and other meteorological parameters.
\end{itemize}

Merging these two datasets is a strategic choice for several reasons. Firstly, it allows for a comprehensive analysis of the relationship between air pollution levels and various weather conditions. By integrating pollutant concentrations with corresponding meteorological data, we can better understand how factors like temperature, wind speed, and precipitation influence air quality. This holistic approach enhances the model's ability to capture the multifaceted nature of air pollution dynamics.

Furthermore, we extended the merged dataset by incorporating precise geographical coordinates for each of the five cities analyzed. Using the Google Maps API, we retrieved the exact latitude and longitude for each city, adding these as additional predictors in our model. Including geographic coordinates is expected to account for spatial dependencies and regional differences in pollution patterns, thereby potentially improving the model's predictive accuracy.

In summary, the final dataset comprises hourly air quality measurements, detailed historical weather data, and geographical information for the mentioned Indian cities. This integrated dataset provides a robust foundation for developing a predictive model aimed at forecasting daily air pollution levels, balancing data comprehensiveness with manageability.

\newpage


% ---------------------------------------------------------------------------------------
% ---------------------------------------------------------------------------------------
% ---------------------------------------------------------------------------------------



\section{Literature Review}
Air pollution has been extensively studied worldwide due to its significant impacts on public health, ecosystems, and the economy. Numerous studies have demonstrated the adverse health effects of air pollutants, including respiratory and cardiovascular diseases, leading to increased morbidity and mortality rates. Particularly, countries like India and China have received considerable attention because of their severe air quality issues, which are exacerbated by rapid industrialization, urbanization, and population growth.

The relationships between air pollution, the greenhouse effect, ozone layer depletion, and climate change are well recognized \citep{manisalidis2020environmental}. Air pollutants such as carbon dioxide, methane, and nitrous oxide trap heat in the atmosphere, which contributes to global warming. Additionally, particles like black carbon can absorb sunlight, further warming the atmosphere. Certain chemicals, including chlorofluorocarbons, are known to damage the ozone layer. When the ozone layer thins, more harmful ultraviolet (UV) radiation reaches the Earth's surface, increasing the risk of skin cancer and cataracts in humans and negatively impacting ecosystems and agriculture. These interconnected issues highlight the importance of addressing air pollution not only for immediate health benefits but also for long-term environmental sustainability.

During our literature review, we encountered a vast number of articles related to our research topic. The widespread interest among scientists and data analysts facilitated the collection of numerous sources. However, the primary challenge lay not in identifying relevant studies, but in thoroughly reading, analyzing, filtering, and synthesizing the information to create a coherent summary. As our research progressed, we found that many studies have attempted to develop statistical and artificial intelligence models to predict air pollution levels, utilizing both global datasets and data specific to particular regions. Researchers have employed a variety of techniques, including multivariate regression, neural networks, and machine learning algorithms, to forecast pollutant concentrations based on diverse predictors such as meteorological conditions, emission sources, and socio-economic factors. These models have been applied at both global and regional scales, providing valuable insights for environmental management and policy-making. Despite this extensive body of research providing a solid foundation for our study, we identified that does not exist studies that have examined the five major Indian cities we are focusing on using integrated datasets.

To effectively manage and synthesize the extensive body of literature, we categorized the scientific articles into two primary groups: general studies on air pollution in India and scientific research focused on developing predictive models for air pollution. The first group is essential because it provides a solid scientific foundation for our study. These general articles enhance our understanding of the environmental, health, and socio-economic factors associated with air pollution in India, which equips us to better analyze data and select relevant predictors for our research. The second group is even more important, as these studies have already attempted to build predictive models similar to ours. By examining their methodologies and findings, we can leverage existing knowledge, refine our approach, and enhance the robustness of our predictive models, thereby building upon a solid foundation of established research.

\subsection{General articles}


\subsection{Existing predicting models}
TODO

\newpage 


\newpage

% ---------------------------------------------------------------------------------------
% ---------------------------------------------------------------------------------------
% ---------------------------------------------------------------------------------------



% Acknowledgements should go at the end, before appendices and references


% Manual newpage inserted to improve layout of sample file - not
% needed in general before appendices/bibliography.

\bibliography{references}

% Note: in this sample, the section number is hard-coded in. Following
% proper LaTeX conventions, it should properly be coded as a reference:

%In this appendix we prove the following theorem from
%Section~\ref{sec:textree-generalization}:



\end{document}
