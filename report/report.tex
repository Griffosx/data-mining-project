\documentclass[twoside,11pt]{article}

% Any additional packages needed should be included after jmlr2e.
% Note that jmlr2e.sty includes epsfig, amssymb, natbib and graphicx,
% and defines many common macros, such as 'proof' and 'example'.
%
% It also sets the bibliographystyle to plainnat; for more information on
% natbib citation styles, see the natbib documentation, a copy of which
% is archived at http://www.jmlr.org/format/natbib.pdf

\usepackage{jmlr2e}
\usepackage{lipsum}
\usepackage{amsmath}
\usepackage{hyperref}
\usepackage{algorithm,algorithmic}

\setcitestyle{square}

% Definitions of handy macros can go here

\newcommand{\dataset}{{\cal D}}
\newcommand{\fracpartial}[2]{\frac{\partial #1}{\partial  #2}}

\renewcommand{\algorithmicrequire}{\textbf{Input:}}
\renewcommand{\algorithmicensure}{\textbf{Output:}}

% Heading arguments are {volume}{year}{pages}{submitted}{published}{author-full-names}
\jmlrheading{}{2024}{}{}{}{Griffon, Smoliakov, Yntykbay}

% Short headings should be running head and authors last names

\ShortHeadings{Predicting Air Pollution Levels in India}{Griffon}
\firstpageno{1}

\begin{document}

\title{Predicting Air Pollution Levels in Five Major Indian Cities}

\author{\name Aleksandr J. Smoliakov \email your.email@stud.mif.vu.lt \\
  \name Danial Yntykbay \email your.email@stud.mif.vu.lt \\
  \name Davide Giuseppe Griffon \email davide.griffon@stud.mif.vu.lt \\
  \addr Data Science study programme\\
       Faculty of Mathematics and Informatics}

\editor{Jurgita Markevi\v{c}i\={u}t\.{e}}

\maketitle


\begin{abstract}

  In 2022, the United Nations recognized the right to a clean environment, including clean air, as a fundamental human right, underscoring the crucial link between environmental quality and human health. Air pollution ranks among the most pressing global health threats, causing an estimated 3 to 9 million deaths annually \citep{owid-data-review-air-pollution-deaths}, with India accounting for a substantial share \citep{Dey2020}. The country exemplifies this crisis, with multiple cities consistently listed among the world’s most polluted. India’s air pollution stems from a combination of natural and anthropogenic sources, emitting harmful pollutants such as carbon monoxide (CO) and particulate matter (PM\textsubscript{10} and PM\textsubscript{2.5}). Rapid expansion in industrial, power, and transportation sectors, coupled with both planned and unplanned urbanization, has significantly elevated ambient air pollution levels across the country \citep{gordon2018air}. The increasing number of automobiles and reliance on coal-based power generation are projected to further deteriorate air quality in the coming decade. Beyond human health implications, air pollution has severe consequences for economic productivity and climate change \citep{Yarlagadda2022}. Addressing this challenge requires a comprehensive understanding of the factors contributing to urban air pollution.
  
  This topic was chosen due to its relevance in addressing the escalating air quality issues in India and the availability of extensive historical weather and air quality data. Previous studies on air quality prediction often focus on pollutants like PM\textsubscript{2.5} and PM\textsubscript{10}. Here, we broaden the scope by analyzing seven different pollutants across five major Indian cities—Bengaluru (Bangalore), Delhi, Hyderabad, Jaipur, and Mumbai—assessing their seasonal trends and interdependencies, and examining a wider set of meteorological and temporal variables. The study also enhances interpretability by comparing feature importance across cities, offering a clearer understanding of regional differences.

  This project aims to develop a predictive model for air pollution levels using a multivariate regression approach. The model will incorporate a wide array of predictors, including geographic location, seasonal variations, meteorological data, temperature, and possibly socio-economic factors. By evaluating various modeling techniques—starting with linear regression and potentially expanding to more complex methods such as Gradient Boosting—and analyzing data correlations, the study seeks to identify the most effective methods for accurately forecasting urban air pollution levels.
  
  Our initial steps involve data preprocessing, exploratory data analysis (EDA), and missing data imputation. EDA will identify trends and possible outliers, while correlation analysis will reveal interdependencies between pollutants and weather variables. This study demonstrates the practical application of data mining techniques using real-world environmental data, serving as a foundation for further exploration in this field.
  
  \end{abstract}

\newpage

% ---------------------------------------------------------------------------------------
% ---------------------------------------------------------------------------------------
% ---------------------------------------------------------------------------------------


\section{Data Sources}

This project utilizes two primary sources of data to analyze air pollution levels across five major Indian cities:

\begin{itemize}
    \item \textbf{Air Quality Data in India}: Available at \href{https://www.kaggle.com/datasets/rohanrao/air-quality-data-in-india}{Kaggle}. This dataset provides hourly measurements of various air pollutants and particulate matter. The data is collected from multiple weather stations located within each of the five major cities. Recognizing these larger cities may have several monitoring stations to capture spatial variability in air quality, we aggregate the pollutant levels by averaging the measurements from all stations within a city for each hour. This averaging process ensures that the data represents the overall air quality of the city rather than isolated monitoring points.

    \item \textbf{Historical Weather Data for Indian Cities}: Available at \href{https://www.kaggle.com/datasets/hiteshsoneji/historical-weather-data-for-indian-cities}{Kaggle}. This dataset includes hourly weather-related features for the same set of cities, encompassing over 20 variables such as precipitation (mm), wind speed, temperature, humidity, and other meteorological parameters. \cite{hitesh_soneji_2020}
\end{itemize}

Merging these two datasets is a strategic choice for several reasons. Firstly, it allows for a comprehensive analysis of the relationship between air pollution levels and various weather conditions. By integrating pollutant concentrations with corresponding meteorological data, we can better understand how factors like temperature, wind speed, and precipitation influence air quality. This holistic approach enhances the model's ability to capture the multifaceted nature of air pollution dynamics.

Furthermore, we extended the merged dataset by incorporating precise geographical coordinates for each of the five cities analyzed. Using the Google Maps API, we retrieved the exact latitude and longitude for each city, adding these as additional predictors in our model. Including geographic coordinates is expected to account for spatial dependencies and regional differences in pollution patterns, thereby potentially improving the model's predictive accuracy.

In summary, the final dataset comprises hourly air quality measurements, detailed historical weather data, and geographical information for the mentioned Indian cities. This integrated dataset provides a robust foundation for developing a predictive model aimed at forecasting daily air pollution levels, balancing data comprehensiveness with manageability.

\newpage


% ---------------------------------------------------------------------------------------
% ---------------------------------------------------------------------------------------
% ---------------------------------------------------------------------------------------



\section{Literature Review}
Air pollution has been extensively studied worldwide due to its significant impacts on public health, ecosystems, and the economy. Numerous studies have demonstrated the adverse health effects of air pollutants, including respiratory and cardiovascular diseases, leading to increased morbidity and mortality rates. Particularly, countries like India and China have received considerable attention because of their severe air quality issues, which are exacerbated by rapid industrialization, urbanization, and population growth.

The relationships between air pollution, the greenhouse effect, ozone layer depletion, and climate change are well recognized \citep{manisalidis2020environmental}. Air pollutants such as carbon dioxide, methane, and nitrous oxide trap heat in the atmosphere, which contributes to global warming. Additionally, particles like black carbon can absorb sunlight, further warming the atmosphere. Certain chemicals, including chlorofluorocarbons, are known to damage the ozone layer. When the ozone layer thins, more harmful ultraviolet (UV) radiation reaches the Earth's surface, increasing the risk of skin cancer and cataracts in humans and negatively impacting ecosystems and agriculture. These interconnected issues highlight the importance of addressing air pollution not only for immediate health benefits but also for long-term environmental sustainability.

During our literature review, we encountered a vast number of articles related to our research topic. The widespread interest among scientists and data analysts facilitated the collection of numerous sources. However, the primary challenge lay not in identifying relevant studies, but in thoroughly reading, analyzing, filtering, and synthesizing the information to create a coherent summary. As our research progressed, we found that many studies have attempted to develop statistical and artificial intelligence models to predict air pollution levels, utilizing both global datasets and data specific to particular regions. Researchers have employed a variety of techniques, including multivariate regression, neural networks, and machine learning algorithms, to forecast pollutant concentrations based on diverse predictors such as meteorological conditions, emission sources, and socio-economic factors. These models have been applied at both global and regional scales, providing valuable insights for environmental management and policy-making. Despite this extensive body of research providing a solid foundation for our study, we identified that does not exist studies that have examined the five major Indian cities we are focusing on using integrated datasets.

To effectively manage and synthesize the extensive body of literature, we have chosen not to include general studies on air pollution in India, as they do not directly contribute to the development of our predictive model. Instead, we have focused our review on two specific categories of research that are more pertinent to our objectives: "Causal and Correlational Studies on Urban Air Pollution" and "Predictive Models". The first category delves into the factors influencing air pollution levels in urban areas, providing valuable insights that inform the selection of variables and the structural framework of our model. The second category encompasses studies that have developed predictive models for air pollution, offering methodologies and approaches that we can build upon to enhance the accuracy and reliability of our own model. By concentrating on these two groups, we aim to leverage existing knowledge effectively and advance our research in a meaningful way.


\subsection{Causal and Correlational Studies on Urban Air Pollution}

Understanding the dynamics of urban air pollution is essential for developing effective strategies to enhance air quality and protect public health. Several studies have focused on analyzing the correlations and underlying causes of air pollution in metropolitan environments, rather than constructing predictive models.

\citep{M2024AirQuality} investigates air pollution across various urban hotspots in Chennai, India. This research assesses hourly concentrations of pollutants such as PM\textsubscript{10}, PM\textsubscript{2.5}, SO\textsubscript{2}, NO\textsubscript{2}, and CO across key areas—industrial, traffic, commercial, and residential zones—over the course of 2022. A key methodological approach employed in this study is the Coefficient of Divergence (COD), which quantifies spatial variations in pollutant concentrations among the different hotspots.

One of the significant findings of this study is the impact of wind on pollution dispersion. When wind speeds are low (0–3 m/s), CO levels tend to be higher, indicating that pollutants are not dispersing effectively and are accumulating near their sources. Conversely, when the wind blows from the south and southeast at moderate speeds (2–6 m/s), the concentrations of PM\textsubscript{2.5} and PM\textsubscript{10} increase. This suggests that pollutants from nearby industries are being transported toward the monitoring stations, highlighting the crucial role of meteorological conditions in air quality.

A similar study was conducted in a different city in India by \citep{Suthar2024Annual}, aiming to identify seasonal patterns and understand how meteorological factors influence pollutant levels. The research included a correlation analysis between air pollutants and meteorological parameters—wind speed (WS), wind direction (WD), relative humidity (RH), and solar radiation (SR). Over three consecutive years, the analysis revealed that WD, WS, and RH generally had a negative correlation with all measured air pollutants. Calm wind conditions inhibit the dispersion of pollutants, resulting in higher concentrations near the ground, underscoring the importance of WS and WD in the dispersion and transport of air pollutants.

Similarly, \citep{refId0} conducted a six-year analysis in Pune, India, assessing the correlations between pollutants and meteorological factors. The study revealed that most pollutants were positively correlated with each other and with temperature, except for O\textsubscript{3}, which had a negative correlation. Wind speed showed a strong negative correlation with pollutant levels, emphasizing its role in pollutant dispersion.

Expanding this line of research to European cities, \citep{Rowland2024} examined the relationship between meteorological parameters and the concentrations of NO\textsubscript{2}, O\textsubscript{3}, PM\textsubscript{10}, and PM\textsubscript{2.5} in Krakow, Paris, and Milan during 2021. The study found that NO\textsubscript{2}, PM\textsubscript{10}, and PM\textsubscript{2.5} concentrations were higher during winter and lower during summer, exhibiting negative correlations with temperature, while O\textsubscript{3} showed the opposite trend. Wind speed was inversely related to particulate matter and NO\textsubscript{2} levels but positively correlated with O\textsubscript{3} concentrations. These findings highlight the influence of meteorological conditions on pollutant levels and the occurrence of the “Ozone weekend effect” in these cities.

These studies collectively underscore the significant impact of meteorological factors on urban air pollution across diverse geographic regions. The consistent observations of pollutant behavior in relation to temperature, wind speed, and other meteorological parameters highlight the necessity of incorporating environmental conditions into air quality management and policy-making.


\subsection{Predictive Models}

Predictive modeling plays a crucial role in understanding and forecasting air pollution levels, which is essential for public health planning and environmental management. Various studies have employed different statistical and machine learning approaches to predict concentrations of air pollutants, utilizing both historical pollution data and meteorological factors.

Singh et al.\ \citep{SINGH2012244} investigated both linear and nonlinear methods for forecasting urban air quality, aiming to improve prediction accuracy in complex urban environments. The study examined the effectiveness of different modeling approaches for predicting concentrations of common urban pollutants such as PM$_{10}$, NO, CO, and O$_3$. Specifically, they applied linear models like multiple linear regression and nonlinear models including artificial neural networks (ANNs) to compare their performance in capturing pollution patterns. The findings indicated that nonlinear models, particularly ANNs, provided better prediction accuracy than linear models, highlighting the importance of nonlinear approaches in modeling air pollution in urban settings.

Sanjeev et al.\ \citep{Sanjeev2021} developed predictive models for air quality using machine learning algorithms, focusing on Artificial Neural Networks (ANN), Support Vector Machines (SVM), and Random Forests (RF). Their study aimed to identify the most efficient algorithm for air quality prediction. The models were evaluated based on accuracy scores, with the RF-based model achieving the highest accuracy of 99.4\%, compared to 93.5\% for SVM and 90.4\% for ANN. The results demonstrate that the Random Forest algorithm is the most effective among the three for predicting air quality.

Kothandaraman et al.\ \citep{Kothandaraman2022Intelligent} focused on predicting PM${2.5}$ pollutant levels by employing a variety of machine learning algorithms, including linear regression, Random Forest, K-Nearest Neighbors, Ridge and Lasso regression, XGBoost, and AdaBoost. Their study utilized historical PM${2.5}$ data and relevant meteorological features such as temperature, humidity, wind speed, and precipitation collected from monitoring stations in Anand Vihar, Delhi, over the period from January 2014 to December 2019. By evaluating the performance of these models through statistical metrics like Mean Absolute Error (MAE), Root Mean Square Error (RMSE), and R-squared ($R^2$), they found that ensemble methods such as XGBoost and Random Forest outperformed other algorithms in terms of predictive accuracy. These results highlight the effectiveness of advanced machine learning techniques in modeling air pollution and the critical role of incorporating meteorological data.

Kumar et al.\ \citep{Kumar2023} tackled the challenging task of air quality prediction by analyzing air pollution data from 23 Indian cities over a six-year period. They undertook comprehensive data preprocessing, which involved handling missing values, addressing outliers, normalizing data, performing feature selection, and applying logarithmic transformations to address skewed features. Exploratory data analysis revealed a significant drop in pollutant levels in 2020, likely due to COVID-19 lockdowns. To address data imbalance, they applied the Synthetic Minority Over-sampling Technique (SMOTE). The dataset was split into training and testing subsets in a 75–25\% ratio. They conducted ML-based AQI prediction using various models, both with and without SMOTE resampling, and presented a comparative analysis. The models were evaluated using standard metrics such as accuracy, precision, recall, F1-score, and statistical error metrics (MAE, RMSE, RMSLE, $R^2$). The XGBoost model emerged as the best performer, achieving the highest accuracy in both training and testing phases, while the SVM model exhibited the lowest accuracy. The Random Forest model also performed relatively well, especially when SMOTE was applied. The study highlights the effectiveness of ensemble learning methods in AQI prediction and suggests that future work could explore deep learning techniques to further enhance prediction accuracy.

Roy et al.\ \citep{ROY20244106} conducted a study in the densely populated northern Indian states of Delhi, Haryana, and Uttar Pradesh, where air pollution is a critical concern. They analyzed PM${2.5}$ concentrations in relation to meteorological factors including maximum and minimum air temperatures, precipitation, surface pressure, and wind. The study applied Ordinary Least Squares (OLS) regression and Geographically Weighted Regression (GWR) to examine the complex relationships between PM${2.5}$ concentrations and various environmental and meteorological parameters across different seasons and geographical locations. Initially, the OLS model was utilized to identify significant predictors of PM$_{2.5}$ concentrations, demonstrating strong correlations with $R^2$ values of 0.93 for summer and 0.94 for winter. Subsequently, GWR was implemented to assess the spatial non-stationarity of the identified relationships, enabling a deeper understanding of how the influence of meteorological and environmental factors varies across different geographical areas. The findings highlighted that GWR provided better model performance by accounting for spatial variability, emphasizing the importance of considering geographical factors in air pollution modeling.


\newpage

% ---------------------------------------------------------------------------------------
% ---------------------------------------------------------------------------------------
% ---------------------------------------------------------------------------------------



% Acknowledgements should go at the end, before appendices and references


% Manual newpage inserted to improve layout of sample file - not
% needed in general before appendices/bibliography.

\bibliography{references}

% Note: in this sample, the section number is hard-coded in. Following
% proper LaTeX conventions, it should properly be coded as a reference:

%In this appendix we prove the following theorem from
%Section~\ref{sec:textree-generalization}:



\end{document}